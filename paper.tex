% SPDX-FileCopyrightText: Copyright (c) 2024-2025 Yegor Bugayenko
% SPDX-License-Identifier: MIT

\documentclass[sigplan,nonacm]{acmart}
\settopmatter{printfolios=false,printccs=false,printacmref=false}
\usepackage{ffcode}
\usepackage{natbib}
\usepackage{doi}
\usepackage{eolang}
\usepackage{href-ul}
\usepackage[capitalize]{cleveref}
\usepackage{to-be-determined}
\usepackage{csquotes}
\usepackage[utf8]{inputenc}
\usepackage[T2A]{fontenc}
\usepackage[russian,english]{babel}
  \DeclareFontFamilySubstitution{T2A}{LinuxBiolinumT-TLF}{LibertinusSans-TLF}
  \renewcommand\ttdefault{cmtt}
% see https://tex.stackexchange.com/a/256190/1449
\usepackage{url}
  \def\UrlBreaks{\do\/\do-}
\usepackage{breakurl}

\setlength{\footskip}{13.0pt}

\title{On the Origin of Objects by Means of Careful Selection}

\author{Yegor Bugayenko}
\orcid{0000-0001-6370-0678}
\email{yegor256@gmail.com}
\affiliation{
  \institution{Huawei}
  \country{Russia}
  \city{Moscow}
}

\author{Maxim Trunnikov}
\orcid{0000-0001-8545-4797}
\email{mtrunnikov@gmail.com}
\affiliation{
  \institution{Huawei}
  \country{Russia}
  \city{Moscow}
}

\ccsdesc[100]{Object-Oriented Programming}
\keywords{Object-Oriented Programming}

\newcommand\aff[1]{\ff{\textcolor{gray}{\(\star\)}#1}}
\newcommand\deff[1]{\ff{\textcolor{blue!50!black}{\textbf{#1}}}}
\newcommand\adeff[1]{\aff{\textcolor{blue!50!black}{\textbf{#1}}}}
\newcommand\eohex[1]{\ff{#1}}

\tolerance=2000
\raggedbottom

\begin{document}

\begin{abstract}
We introduce a taxonomy of objects for the \eolang{} programming language.
This taxonomy is designed with a few principles in mind: non-redundancy and simplicity.
The taxonomy is supposed to be used as a navigation map by \eolang{} programmers.
It may also be helpful as a guideline for designers of other object-oriented languages or libraries for them.
\end{abstract}

\maketitle

\section{Introduction}

Each object-oriented programming language offers its own set of objects, classes, types, functions, constants, traits, templates, and so on, which programmers can use off-the-shelf to model their specific use cases:
``Development Kit'' in Java~\citep{jdk2024,jdk8},
``Standard Library'' in
  C++~\citep{cpp2024,josuttis2012cpp},
  Python~\citep{python2024,hellmann2017python},
  Swift~\citep{swift2024,deitel2015swift},
  Rust~\citep{rust2024,blandy2021rust},
  and
  Ruby,
``.NET Standard'' in C\#~\citep{net2023,abrams2005net},
and ``Built-in Objects'' in JavaScript~\citep{js2024,crockford2008js}.

Each standard library (SL) freely defines its own principles of abstraction.
For example, to get the absolute value of a numeric object \ff{x} in Java, one has to call a static method of another class \ff{Math.abs(x)}, while in Ruby it would be a property of the same object \ff{x.abs}.
For example, in Java, an attempt to get a value from a hash map by a key will produce \ff{null} in case of its absence, while in C++ it will return an empty iterator.
There are many similar examples demonstrating the absence of a common ground between SLs.

Earlier, \citet{booch1990design} suggested their own components for OOP.
We suggest our own taxonomy of objects for \eolang{}\footnote{\url{https://www.eolang.org}}, a strictly formal~\citep{kudasov2022formalizing} object-oriented programming language with an intentionally reduced feature set~\citep{bugayenko2021eolang}.
All objects are grouped as such:

\begin{itemize}
    \item Bytes (\cref{sec:bytes})
    \item Primitives (\cref{sec:primitives})
    \item Tuple (\cref{sec:tuple})
    \item Error handling (\cref{sec:errors})
    \item Flow control abstractions (\cref{sec:flow})
    \item Non-standard bit-width numbers (\cref{sec:digits})
    \item Memory abstractions (\cref{sec:memory})
    \item Math functions and algorithms (\cref{sec:math})
    \item String manipulations (\cref{sec:strings})
    \item Structs (\cref{sec:structs})
    \item I/O Streams (\cref{sec:streams})
    \item File System abstractions (\cref{sec:fs})
    \item Net Sockets (\cref{sec:sockets})
\end{itemize}

An object in \eolang{} is either a composition of other objects or an atom, which is a foreign function interface to a lower-level runtime such as, for example, JVM or Assembly.
In this paper, we denote atoms by a star, for example: \aff{times}.

Throughout the paper, we use dark blue color to highlight an object when it is mentioned for the first time.

In this paper\footnote{%
\LaTeX{} sources of this paper are maintained in the
\href{https://github.com/REPOSITORY}{REPOSITORY} GitHub repository,
the rendered version is \href{https://github.com/REPOSITORY/releases/tag/0.0.0}{\ff{0.0.0}}.}
we don't provide exact specification of each object, in order to avoid conflicts with their exact definitions in the Objectionary\footnote{\url{https://github.com/objectionary/home}}.

\section{Principles}

We adhere to a few principles:
\begin{enumerate}
  \item We tend to minimize the global scope, keeping the number of global objects to the absolute possible minimum.
  \item Wherever possible, objects must be implemented in \eolang{}, instead of being atoms.
  \item There should be no functionality implemented two or more times in the entire taxonomy.
  \item Object names must not be abbreviated and must constitute an English sentence when being used through dot notation, for example:
\begin{ffcode}
x.times y > z
\end{ffcode}
means ``\(x\) times \(y\) is \(z\).''
There are a few exceptions though, like \ff{eq}, \ff{seq}, \ff{lt}, and a few others.
\end{enumerate}

In case of any runtime error, an object is supposed to ``return'' an error
object, with the \ff{message} attribute present (attached to any object) and
the \ff{@} attribute absent.

\section{Bytes}\label{sec:bytes}

The center of the taxonomy is the \deff{bytes} data object, which is an abstraction
of a sequence of bytes. The sequence can either be empty or contain a
theoretically unlimited number of bytes. A byte is a unit of information that
consists of eight adjacent binary digits (bits), each of which consists of a \ff{0}
or \ff{1}. In \eolang{} syntax, a sequence of bytes is denoted as either \eohex{---}
(double dash) for an empty one or as a concatenation of \eohex{xx-}, where
\ff{xx} is a hexadecimal representation of a byte. For example, the \ff{2A-} is
a one-byte sequence, where the only byte equals to the number 42.

There are also attributes for bitwise operations:
\adeff{and},
\adeff{or},
\adeff{xor},
\adeff{not},
\adeff{right}
and
\deff{left}.

The object also has the \adeff{eq} attribute for byte-by-byte comparing
itself with another sequence of bytes.

The \adeff{size} attribute is the total number of bytes in the sequence.

The \adeff{slice} attribute represents a subsequence inside the current one.

The \adeff{concat} attribute is a new sequence, which is a concatenation
of two byte sequences: the current and the provided one.

\section{Primitives}\label{sec:primitives}

There are four primitive data objects described below.
All of them have an \adeff{as-bytes} attribute, which is an abstraction of their representation as a sequence of bytes.
All primitives have an \deff{eq} attribute, which is \ff{true} if the primitive is equal to another object, through the use of \ff{\^{}.as-bytes.eq}.

\subsection{Boolean}

There are two pre-defined objects \deff{true} and \deff{false} which decorate \eohex{01-} and \eohex{00-} accordingly.

The \deff{not} attribute is a Boolean inversion.
Attributes \deff{and}, \deff{or} are logical conjunction, disjunction of the object itself with other object.

\subsection{Number}

The object \deff{number} is an abstraction of a floating point number, which takes eight bytes and fills them up according to the format suggested by \citet{ieee754}.

The \adeff{plus} attribute is a sum of this number and another \ff{number}.
The \deff{minus} attribute is the subtraction of another \ff{number} from this number.
The \adeff{times} attribute and \adeff{div} are multiplication and division of this number and another \ff{number}.
The \deff{neg} attribute is the same number with a different sign.
The \deff{floor} attribute is a rounded down \ff{number}.

The \adeff{gt} attribute is a boolean object \ff{true} if it is ``greater than'' another \ff{number} and \ff{false} otherwise.
Attributes \deff{lt}, \deff{lte}, and \deff{gte} are ``greater than,'' ``less than or equal,'' and ``greater than or equal'' comparisons respectively.

Floating-point numbers with a different bit-width are explained in \cref{sec:digits}.
Other manipulations with floating-point numbers can be done in their decorators.
Some of them are explained in \cref{sec:math}.

\subsection{String}

The \deff{string} object is an abstraction of a piece of Unicode text in UTF-8 encoding, according to \citet{unicode}.
UTF-8 uses one byte for the first 128 code points, and up to 4 bytes for other characters.

The \adeff{slice} attribute takes a piece of a string as another \ff{string}.
The \adeff{length} attribute is a total count of characters in the string.

Strings in other encodings, such as UTF-16 or CP1251, may be implemented through direct manipulations with \ff{bytes}.

Other manipulations with strings are explained in \cref{sec:strings}.

\section{Tuple}\label{sec:tuple}

The \deff{tuple} object is an abstraction of an immutable sequence of objects.
For example, this code represents a tuple \ff{x} of numbers and strings:

\begin{ffcode}
* > x
  * "green" 0x0D1
  * "red" 0xF21
  * "blue" 0x11C
\end{ffcode}

The \deff{with} attribute is a new \ff{tuple} with all elements from the current tuple and a new element added to the end of it.
%
The \deff{at} attribute is the element of the tuple at the specified index---positive indices are zero-based starting from the left, and negative indices start at -1 from the right.
%
The \deff{length} attribute is the total number of elements in the tuple.

Other manipulations with tuples can be implemented in their
decorators. Some of them are explained in \cref{sec:structs}.

\section{Error Handling}\label{sec:errors}

Here we define objects that help handle errors and exceptional situations.

The \adeff{error} object causes program termination at the first attempt to dataize it.
It encapsulates any other object, which can play the role of an exception that is floating to the upper level:

\begin{ffcode}
[x] > d
  if. > @
    x.eq 0
    error "Can't divide by zero"
    42.div x
\end{ffcode}

The \adeff{try} object enables catching of \aff{error} objects and extracting exceptions from them.
For example, the following code prints ``division by zero'' and then ``finally'':

\begin{ffcode}
try
  42.div 0
  QQ.io.stdout e > [e]
  []
    QQ.io.stdout "finally" > @
\end{ffcode}

\section{Flow Control}\label{sec:flow}

Here we define objects that help control dataization flow.

The \deff{seq} object is an abstraction of sequence of objects to be dataized sequentially.
For example, this code prints one message to the console and then terminates the program due to
the inability to dataize the division by zero in the middle:

\begin{ffcode}
seq *
  QQ.io.stdout "Hello, world!"
  42.div 0
  QQ.io.stdout "Bye!"
\end{ffcode}

The objects \ff{true} and \ff{false} both have an attribute \deff{if} which is a branching mechanism,
expecting two arguments: a ``positive'' object, and a ``negative'' object. When
being dataized, the object \ff{true.if} always returns the ``positive'' object,
the object \ff{false.if} always returns the ``negative'' object.
This code gives a title to a random number:

\begin{ffcode}
QQ.math.random.pseudo > r
QQ.io.stdout
  QQ.txt.sprintf *1
    "Coin toss: %s"
    if.
      r.gte 0.5
      "heads"
      "tails"
\end{ffcode}

The \deff{while} object is an iterating mechanism, expecting two arguments.
Both are abstract objects with one free attribute for current iteration index.
The first argument is a condition, the second one is the body.
When being dataized, the object \ff{while} repeatedly dataizes the body while the condition is \ff{true}.
In the end, it returns the body or \ff{false} if no iterations happened.
This code makes a few attempts to find a random number that is smaller than 0.1:

\begin{ffcode}
QQ.math.random.pseudo > r
while
  [i] >>
    r.gte 0.1 > @
  [i]
    QQ.io.stdout > @
      QQ.txt.sprintf *1
        "Attempt #%d"
        i
\end{ffcode}

The \deff{go} object enables forward and backward ``jumps'' either immediately finishing dataization or restarting it.
For example, this code won't print a message to the console if \ff{x} is equal to zero:

\begin{ffcode}
go.to
  [g]
    seq * > @
      if > y
        x.eq 0
        g.forward 0
        42.div x
      QQ.io.stdout
        QQ.txt.sprintf *1
          "42/x = %d"
          y
\end{ffcode}

In this example, the number will be read from the console again, if it is equal to zero:

\begin{ffcode}
go.to
  [g]
    seq * > @
      at. > x
        QQ.txt.sscanf
          "%d"
          QQ.io.stdin.next-line
        0
      if
        x.eq 0
        g.backward
        QQ.io.stdout
          QQ.txt.sprintf
            "Number %d is OK!"
            * x
\end{ffcode}

The \deff{switch} object is an abstraction of multi-case branching that encapsulates \((k, v)\) pairs and the
dataization result is equal to \(v\) of the first pair where \(k\) is \ff{true}, while the last pair is the one
to be used if no other pairs match:

\begin{ffcode}
QQ.io.stdout
  switch *
    *
      password.eq "swordfish"
      "password is correct!"
    *
      password.eq ""
      "empty password is not allowed"
    *
      false
      "password is wrong"
\end{ffcode}

\section{Digits}\label{sec:digits}

Here we define objects that represent numbers with smaller or larger bitwise width compared with the default 64-bit convention.

The \deff{i16}, \deff{i32}, and \deff{i64} objects are decorators of \ff{bytes} with a predefined size.
They implement the same numeric operations as \ff{number}.

\section{Memory}\label{sec:memory}

Here we define objects that manage memory.

The \deff{malloc} object is an abstraction of a random-access memory of certain size.
It behaves like the \ff{bytes}, but has an additional \adeff{write} attribute.
In this example, one kilobyte of memory with random data is allocated when a copy of \ff{malloc} is created,
six bytes are written starting at the 200th byte, and then a string is read back and \ff{"Hello!"} is printed:

\begin{ffcode}
malloc.of
  1024
  [m]
    seq * > @
      m.write 200 "Hello!"
      QQ.io.stdout
        as-string.
          m.read 200 6
\end{ffcode}

The \ff{malloc} object also supports resizing via the \adeff{resized} attribute and copying the data within
the allocated memory block via the \deff{copy} attribute. In the next example, \ff{malloc} is allocated and filled with
5 bytes, then resized to 10 bytes and its content is duplicated. The result block contains \ff{"hellohello"}.

\begin{ffcode}
malloc.of
  5
  [m]
    seq * > @
      m.write 5 "hello"
      m.resized 10
      m.copy 0 5 5
\end{ffcode}

The result of dataization of \ff{malloc} is the result of dataization of its second argument, which is the scope
where the memory block is allocated and cleared automatically when dataization has happened.

\section{Math}\label{sec:math}

Here we define objects that represent math functions and algorithms.
All objects in this Section belong to the \ff{QQ.math} package.
%
The \deff{random} object is an abstraction of a random \ff{number} between zero and one inclusive.
It has an additional attribute \deff{pseudo} that provides a pseudorandom number generator with a seed.

The \deff{angle} object is a decorator of \ff{number}.
It assumes that the angle is in radians and has the following attributes for trigonometric functions: \adeff{sin},
\adeff{cos}, \deff{tan} and \deff{ctan}.
There are \deff{in-degrees} and \deff{in-radians} attributes for converting from radians to degrees and vice versa.

The \deff{real} object is a decorator of \ff{number} with the following additional attributes:

\begin{itemize}
    \item \deff{exp}: Euler's number raised to the power of \ff{real}
    \item \deff{mod}: modulo from division of itself by another number
    \item \deff{abs}: absolute value of itself (\(\rho\))
    \item \adeff{sqrt}: square root of itself (\(\sqrt{\rho}\))
    \item \adeff{pow}: itself raised to the power (\(\rho^x\))
    \item \adeff{ln}: natural logarithm (\(\ln \rho\))
    \item \adeff{acos}: arccosine of itself
    \item \adeff{asin}: arcsine of itself

\end{itemize}

\section{Strings}\label{sec:strings}

All objects presented in this Section belong to the \ff{QQ.txt} package.

The \deff{sscanf} object encapsulates two \ff{string} objects: a format and a content.
It behaves like a tuple of scanned data objects.
For example, this code parses a hexadecimal number from the console:

\begin{ffcode}
at.
  QQ.txt.sscanf
    "%x"
    QQ.io.stdin
  0
\end{ffcode}

The \deff{sprintf} object builds a string according to the format provided and compliant with \citet[Chapter 5]{posix}, for example:

\begin{ffcode}
QQ.txt.sprintf *1
  "Hi, %s! The weight is %0.2f."
  "Jeff"
  3.14
\end{ffcode}

The object \deff{text} is a decorator of a \ff{string}. It has the following attributes:

\begin{itemize}
    \item \deff{trimmed}: removes both leading and trailing spaces
    \item \deff{trimmed-left}: removes leading spaces
    \item \deff{trimmed-right}: removes trailing spaces
    \item \deff{joined}: joins a tuple with this one as a glue
    \item \deff{repeated}: repeats itself a given number of times
    \item \deff{contains}: checks whether it contains \(x\)
    \item \deff{starts-with}: checks whether it starts with \(x\)
    \item \deff{ends-with}: checks whether it ends with \(x\)
    \item \deff{index-of}: finds the first occurrence of a substring
    \item \deff{last-index-of}: finds the last occurrence of a substring
    \item \deff{up-cased}: makes it upper case
    \item \deff{low-cased}: makes it lower case
    \item \deff{at}: retrieves symbol by given index
    \item \deff{replaced}: finds and replaces a substring
    \item \deff{as-number}: converts itself to \ff{number}
    \item \deff{split}: breaks it into a tuple of strings
    \item \deff{chained}: concatenates two strings
\end{itemize}

The \deff{regex} object is an abstraction of a regular expression, in accordance with \citet[Part~2]{posix}
(Extended Regular Expressions) and full support of Unicode.
The \deff{match} attribute is a split of the \ff{string} into parts.
The \deff{matches} attribute is \ff{true} if the given text matches against the given regex pattern.

\begin{ffcode}
QQ.txt.regex "/((*@и@*)([^ ]+))/i" > r
r.match > m
  "(*@Из искры возгорится пламя@*)"
eq.
  m
  *
    *
      "(*@Из@*)"
      0
      *
        "(*@И@*)"
        * "(*@з@*)" 1 *
    * " "
    *
      "(*@искры@*)"
      3
      *
        "(*@и@*)"
        * "(*@скры@*)" 4 *
    * " (*@возгор@*)"
    *
      "(*@ится@*)"
      15
      *
        "(*@и@*)"
        "(*@тся@*)" 16 *
    * " (*@пламя@*)"
\end{ffcode}

\section{Structs}\label{sec:structs}

Here, we discuss abstractions of lists, maps, sets, and other ``structs.''
All objects presented in this Section belong to the \ff{QQ.structs} package.

\subsection{List}

The object \deff{list} is a decorator of \ff{tuple}.
%
The attribute \deff{is-empty} is \ff{true} if the length of the tuple is zero.
%
The attribute \deff{eq} is \ff{true} if each element of the tuple is equal to the corresponding element of another tuple and the lengths of both tuples are the same.
%
The attribute \deff{with} is a new tuple with new element appended to the end of the list.
%
The attribute \deff{withi} is a new tuple with a new \(i\)-th element appended.
%
The attribute \deff{without} is a new tuple with all elements equal to the given \(item\) removed.
%
The attribute \deff{withouti} is a new tuple with the \(i\)-th element removed.
%
The attributes \deff{each}, \deff{reduced}, \deff{contains}, \deff{filtered}, and \deff{mapped} are respectively similar to \ff{forEach}, \ff{reduce},
\ff{find}, \ff{filter}, and \ff{map} methods of \ff{Array} object in JavaScript~\citep{EcmaScript}.
A few ``twin'' attributes \deff{eachi}, \deff{reducedi}, \deff{filteredi} and \deff{mappedi} are semantically the same, but with an extra \ff{number} argument as a counter of a cycle.
%
The \deff{head} attribute is the first \(N\) items of the list.
%
The \deff{tail} attribute is the last \(N\) items of the list.
%
The \deff{sorted} attribute is a new tuple with elements sorted using the \ff{lt} attribute of the elements passed.
%
The \deff{concat} attribute is a new tuple that concatenates the current tuple with the tuple provided as an argument.

\subsection{Map}

The object \deff{map} is a decorator of \ff{tuple} of \deff{map.entry} objects which are pairs \((k, v)\).
The \ff{map} ensures that all \(k\) are always unique.
It's expected that each \(k\) is dataizable so it's possible to calculate \deff{hash-code-of} it.
%
The attributes \ff{with} and \ff{without} are reimplemented in \ff{map}.
%
The \ff{found} attribute tries to find an object in the map by the given key. The returned
object has two attributes: \deff{exists} which is \ff{true} if the object was found;
and \deff{get} which is the found object or \ff{error}.
%
The \deff{keys} attribute is the \ff{list} of map keys.
%
The \deff{values} attribute is the \ff{list} of map values.
%
The \deff{size} attribute is the \ff{size} of the map.
%
The \deff{has} attribute is \ff{true} if the map contains the given \(key\).

\subsection{Set}

The object \deff{set} is a decorator of \ff{list}.
The \ff{set} ensures that all elements in the list are always unique.
It's expected that each element is dataizable so it's possible to calculate \deff{hash-code-of} it.
%
The attributes \ff{with}, \ff{without} are reimplemented in \ff{set}.

\subsection{Hash code}

The object \ff{hash-code-of} is the pseudo-unique floating-point numeric representation of an object.
It's expected that the given object is dataizable.

\subsection{Ranges}

The object \deff{range} is a \ff{list} that contains a range of elements. It accepts two attributes:
\deff{start} and \deff{end}. The attribute \ff{start} must be an abstract object that must have an
attribute \ff{next} to get the next element and an attribute \ff{lt} to compare with the \ff{end} object.
Every next element also must have attributes \ff{next} and \ff{lt}.
The first object in the chain must have a default value.

The object \deff{range-of-ints} is a \ff{range} of integer numbers from \ff{start} to \ff{end} (soft border) with step = \(1\).
It also accepts two void attributes \ff{start} and \ff{end} which must be integer \ff{number}s. If they're not, an \ff{error} is returned.

\section{I/O Streams}\label{sec:streams}

Here we define manipulations with input and output streams.
All objects presented in this Section belong to the \ff{QQ.io} package.

It is expected that an input stream has the following interface:

\begin{ffcode}
[] > input
  [size] > read /bytes
\end{ffcode}

It is expected that an output stream has the following interface:

\begin{ffcode}
[] > output
  [bytes] > write /true
\end{ffcode}

The \deff{console} object is a basic I/O object that allows interaction with the operating system console. It behaves as ``input'' and ``output.''
%
The \deff{stdout} object is an ``output'' that prints a \ff{string} to the standard output stream.
%
The \deff{stdin} object is an ``input'' and an abstraction of a \ff{string} currently available in the standard input stream.
The attribute \deff{next-line} reads a line until the EOL character (\ff{\char`\\n}).
The attribute \deff{all-lines} reads all lines as long as possible.
If there is no string in the stream, the object blocks dataization and waits.
This code endlessly reads strings from the console and immediately prints them back:

\begin{ffcode}
while
  true > [i]
  [i]
    QQ.io.stdout > @
      QQ.io.stdin.next-line
\end{ffcode}

The object \adeff{bytes-as-input} is an ``input'' created from \ff{bytes}.
%
The object \adeff{malloc-as-output} is an ``output'' directed towards a copy of \ff{malloc}.
%
The object \adeff{tee-input} is an ``input'' and a channel between an ``input'' and an ``output,'' which moves all
available bytes from the former to the latter when being dataized.
%
For example, this code writes text to a temporary file:

\begin{ffcode}
(QQ.fs.file "/tmp/foo.txt").open
  [f]
    QQ.io.tee-input
      QQ.io.bytes-as-input
        "Hello, world!".as-bytes
      f
\end{ffcode}

The object \deff{input-length} is an object which reads all the bytes from the provided ``input'' and returns its length.
%
The objects \deff{dead-input} and \deff{dead-output} are stubs which read from and write to nothing.

\section{File System}\label{sec:fs}

Here we define manipulations with files and directories.
All objects presented in this Section belong to \ff{QQ.fs} package.

The object \deff{path} is an abstraction of file path and is a decorator of \ff{string}.
%
The attribute \deff{separator} is a system-dependent default name-separator character.
%
The attribute \deff{joined} is a utility object that joins the given \ff{tuple} of paths with the current OS separator
and normalizes the result path.
%
The attribute \deff{as-file} is a \deff{file} created from the path itself.
%
The attribute \deff{as-dir} is a \deff{dir} created from the path itself.
%
The attribute \deff{is-absolute} is \ff{true} if the path is absolute and \ff{false} otherwise.
%
The attribute \deff{normalized} is a new \ff{path} with normalized \(uri\). Normalization includes
converting multiple slashes into a single slash and resolving ``.'' (current directory) and ``..''
(parent directory) segments.
%
The attribute \deff{resolved} is a new \ff{path} with a suffix appended to the current one.
%
The attribute \deff{basename} is the name of the file in the path.
%
The attribute \deff{extname} is the extension in the path.
%
The attribute \adeff{dirname} is the name of the directory in the path.

The object \deff{file} is an abstraction of a file and is a decorator of \ff{path}, which is the path of the file.
%
The attribute \deff{as-path} is a \ff{path} of the file.
%
The attribute \adeff{is-directory} is \ff{true} if the file is a directory.
%
The attribute \adeff{exists} is \ff{true} if the file exists.
%
The attribute \deff{touched} makes sure the file exists.
%
The attribute \deff{deleted} removes the file.
%
The attribute \adeff{size} is the size of the file in bytes.
%
The attribute \deff{moved} renames or moves the file to a new place.
%
The attribute \deff{open} opens the file. The \ff{open} object accepts two attributes: file open mode
which is a \ff{string} and file scope. The file scope attribute must be an abstract object
with one void attribute which is \deff{file-stream}. It has the attributes \ff{read} and \ff{write}
which allow working with the file as ``input'' and ``output''. The result of dataization of \ff{file.open} is
the result of dataization of its second argument ``scope''. The file descriptor is closed automatically
when the scope is dataized.

The object \deff{dir} is an abstraction of a directory.
%
The attribute \deff{made} creates the directory with all its parent directories.
%
The attribute \deff{deleted} deletes the directory recursively with all its subdirectories.
%
The attribute \deff{walk} finds files in the directory using glob pattern matching.
%
The attribute \deff{tmpfile} is a temporary file in the directory.

The object \adeff{tmpdir} is a system \ff{directory} for temporary files.

The object \deff{tmpdir} is an abstraction of a temporary directory and is a decorator of \deff{dir}.

In the next example, a temporary file is created and filled up with \ff{"Hello, world"}.
Then the data is read from the file and written to \ff{malloc}.

\begin{ffcode}
QQ.malloc.of
  12
  [m]
    seq > @
      *
        tmpdir
        .tmpfile
        .open
          "w"
          f.write "Hello, world" > [f]
        .open
          "r"
          [f] >>
            m.put > @
              f.read 12
        m
\end{ffcode}

\section{Net}\label{sec:sockets}

Here we define objects used for both server-side and client-side TCP sockets in accordance with \citet{posix}.
All objects presented in this Section belong to the \ff{QQ.net} package.

The \deff{socket} object is an abstraction of a TCP socket.
The \deff{connect} attribute establishes a connection. It accepts an abstract object with one void attribute which is the socket descriptor.
The socket descriptor has the following attributes:

\begin{itemize}
    \item \deff{send}: send a message to the socket
    \item \deff{recv}: receive \(x\) amount of bytes from the socket
    \item \ff{as-input}: make ``input'' from socket
    \item \ff{as-output}: make ``output'' from socket
\end{itemize}

When you dataize \ff{socket.connect}, it creates a new socket, connects to the given
IP on the given port, dataizes the provided scope, closes the socket and returns bytes
retrieved from scope dataization. An \ff{error} is returned if any of the described steps failed.

This code opens a connection to TCP port 80 of \ff{localhost} and then reads the entire stream as a Unicode string:

\begin{ffcode}
(QQ.net.socket "127.0.0.1" 80).connect
  [s]
    QQ.io.tee-input > @
      s.as-input
      QQ.io.console
\end{ffcode}

The \adeff{listen} attribute allows you to listen to incoming connections as a server.
It accepts an abstract object with one void attribute which is the socket descriptor. The current socket descriptor has the
same attributes as the descriptor from \ff{socket.connect} and also one extra attribute \deff{accept}. This attribute is
used to wait for an incoming connection. It also accepts an abstract object with one void attribute which is the client socket descriptor,
which was connected to your server.

This code emulates a server on TCP port 8080 of \ff{localhost}, then it accepts an incoming connection and sends \ff{"Hello, world"} to it:

\begin{ffcode}
(QQ.net.socket "127.0.0.1" 8080).listen
  [s]
    s.accept > @
      [client]
        client.send "Hello, world" > @
\end{ffcode}

\raggedright
\bibliographystyle{ACM-Reference-Format}
\bibliography{bibliography/main}

\end{document}
